\documentclass[boxes]{homework}

% This is a slightly-more-than-minimal document that uses the homework class.
% See the README at http://git.io/vZWL0 for complete documentation.

\name{傅申 PB20000051}        % Replace (Your Name) with your name.
\term{2022 春}     % Replace (Current Term) with the current term.
\course{数理逻辑基础}    % Replace (Course Name) with the course name.
\hwnum{6}          % Replace (Number) with the number of the homework.
\hwname{作业}    
\problemname{练习}    
\solutionname{解:}

% Load any other packages you need here.
\usepackage[
    a4paper,
    top = 2.22cm,
    bottom = 2.22cm,
    left = 1.91cm,
    right = 1.91cm,
    includeheadfoot
]{geometry}
\fancyfootoffset{0pt} % make fancyhdr work properly
\usepackage{ctex}

\begin{document}
\setlength\abovedisplayskip{.125em}
\setlength\belowdisplayskip{.125em}
\problemnumber{14}
\begin{problem}
2. 在以下公式中, 哪些 $x_1$ 的出现是自由的? 哪些 $x_1$ 的出现是约束的? 项 $f_1^2(x_1, x_3)$ 对这些公式中的 $x_2$ 是不是自由的?
\begin{parts}[n]
    \setcounter{enumi}{2}
    \part \label{14.2.3}
    $\forall x_1 R_1^1(x_1)\to \forall x_2R_1^2(x_1, x_2)$
    \part \label{14.2.4}
    $\forall x_2 R_1^2(f_1^2(x_1, x_2), x_1)\to \forall x_1R_2^2(x_3, f_2^2(x_1, x_2))$
\end{parts}
\end{problem}

\begin{solution}
\ref{14.2.3} 第 3 个 $x_1$ 的出现是自由的, 第 1, 2 个 $x_1$ 的出现是约束的.

因为 $x_2$ 在公式中不自由出现, 所以项 $f_1^2(x_1, x_3)$ 对公式中的 $x_2$ 是自由的.

\ref{14.2.4} 第 1, 2 个 $x_1$ 的出现是自由的, 第 3, 4 个 $x_1$ 的出现是约束的.

项 $f_1^2(x_1, x_3)$ 对公式中的 $x_2$ 是不自由的, 因为替换第二个 $x_2$ 后项 $f_1^2(x_1, x_3)$ 中的 $x_1$ 受到约束.
\end{solution}

\problemnumber{14}
\begin{problem}
    3. 设 $t$ 是项 $f_1^2(x_1, x_3)$, $p(x_1)$ 是下面的公式. 确定 $t$ 对 $p(x_1)$ 中的 $x_1$ 是否自由? 如果是自由的, 写出 $p(t)$.
    \begin{parts}[n]
        \setcounter{enumi}{2}
        \part\label{14.3.3}
        $\forall x_2 R_1^1(f_1^1(x_2))\to \forall x_3 R_1^3(x_1, x_2, x_3)$
        \part\label{14.3.4}
        $\forall x_2 R_1^3(x_1, f_1^1(x_1), x_2)\to \forall x_3 R_1^1(f_1^2(x_1, x_3))$
    \end{parts}
\end{problem}
\begin{solution}
    \ref{14.3.3} $t$ 对 $p(x_1)$ 中的 $x_1$ 不自由, 因为替换后 $t$ 中的 $x_3$ 受到约束.

    \ref{14.3.4} $t$ 对 $p(x_1)$ 中的 $x_1$ 不自由, 因为替换后 $t$ 中的 $x_3$ 受到约束.
\end{solution}
\problemnumber{14}
\begin{problem}
    5. 设个体变元 $x$ 在公式 $p(x)$ 中自由出现, 个体变元 $y$ 不在公式 $p(x)$ 中自由出现. 试证, 如果 $y$ 对 $p(x)$ 中的 $x$ 是自由的, 那么 $x$ 对 $p(y)$ 中的 $y$ 也是自由的.
\end{problem}
\begin{solution}
    因为 $x$ 在公式 $p(x)$ 中自由出现, 所以所有的 $x$ 都不是在 $\forall x$ 中或在 $\forall x$ 的范围中. 将 $p(y)$ 中的 $y$ 分为两部分:
    \begin{enumerate}[label = (\alph*), parsep = 0pt, itemsep = 0pt, topsep = .25em]
        \item $p(x)$ 中原有的 $y$, 它们本身就是不自由出现的.
        \item $p(x)$ 中 $x$ 被替换后的 $y$, 因为 $y$ 对 $p(x)$ 中的 $x$ 是自由的, 所以这部分 $y$ 都是自由的, 即它们都不是在 $\forall y$ 中或在 $\forall y$ 的范围中.
    \end{enumerate}
    而因为 $p(x)$ 中的 $x$ 都不是在 $\forall x$ 中或在 $\forall x$ 的范围中, 所以 $p(y)$ 中自由的 $y$ (即 b 部分) 都不是在 $\forall x$ 中或在 $\forall x$ 的范围中. 因此, 用 $x$ 替换 $p(y)$ 中自由出现的 $y$ 后, 这些 $x$ 都不会出现在 $\forall x$ 中或在 $\forall x$ 的范围中. 因此, $x$ 对 $p(y)$ 中的 $y$ 也是自由的.
\end{solution}.
\problemnumber{15}
\begin{problem}
    2. 试证对任意公式 $p$ 与 $q$, 有
    $$
        \vdash \forall x(p\to q)\to (\forall xp\to \forall xq)
    $$
\end{problem}
\begin{solution}
    先证明 $\{\forall x(p\to q), \forall x p\}\vdash \forall xq$:
    \begin{enumerate}[label = (\arabic*), parsep = 0pt, itemsep = 0pt, topsep = .25em]
        \item $\forall x(p\to q)$\hfill 假定
        \item $\forall x(p\to q)\to (p\to q)$\hfill (K4)
        \item $p\to q$ \hfill (1), (2), MP
        \item $\forall x p$\hfill 假定
        \item $\forall x p\to p$\hfill (K4)
        \item $p$\hfill (4), (5), MP
        \item $q$\hfill (6), (3), MP
        \item $\forall xp$\hfill (7), Gen
    \end{enumerate}
    在上面的证明中, 除了 $x$ 外没有使用其他的 Gen 变元, 而 $x$ 显然不在 $\forall x p$ 和 $\forall x(p\to q)$ 中自由出现, 由演绎定理, 先后有 $\{\forall x(p\to q)\}\vdash \forall xp\to \forall xq$, $\vdash \forall x(p\to q)\to (\forall xp\to \forall xq)$. 公式得证.
\end{solution}
\problemnumber{15}
\begin{problem}
    3. 求证:
    \begin{parts}[n]
        \part\label{15.3.1}
        $\{\forall x_1\forall x_2 R_1^2(x_1, x_2)\}\vdash \forall x_1 R_1^2(x_1, x_1)$
        \part\label{15.3.2}
        $\{\forall x_1\forall x_2 R_1^2(x_1, x_2)\}\vdash \forall x_2\forall x_3 R_1^2(x_2, x_3)$
    \end{parts}
\end{problem}
\begin{solution}
    \ref{15.3.1} 证明如下:
    \begin{enumerate}[label = (\arabic*), parsep = 0pt, itemsep = 0pt, topsep = .25em]
        \item $\forall x_1\forall x_2 R_1^2(x_1, x_2)$ \hfill 假定
        \item $\forall x_1\forall x_2 R_1^2(x_1, x_2)\to \forall x_2 R_1^2(x_1, x_2)$\hfill (K4)
        \item $\forall x_2 R_1^2(x_1, x_2)$\hfill (1), (2), MP
        \item $\forall x_2 R_1^2(x_1, x_2)\to R_1^2(x_1, x_1)$\hfill (K4)
        \item $R_1^2(x_1, x_1)$\hfill (3), (4), MP
        \item $\forall x_1 R_1^2(x_1, x_1)$\hfill (5), Gen
    \end{enumerate}
    \ref{15.3.2} 证明如下:
    \begin{enumerate}[label = (\arabic*), parsep = 0pt, itemsep = 0pt, topsep = .25em]
        \item $\forall x_1\forall x_2 R_1^2(x_1, x_2)$ \hfill 假定
        \item $\forall x_1\forall x_2 R_1^2(x_1, x_2)\to \forall x_2 R_1^2(x_1, x_2)$\hfill (K4)
        \item $\forall x_2 R_1^2(x_1, x_2)$\hfill (1), (2), MP
        \item $\forall x_2 R_1^2(x_1, x_2)\to R_1^2(x_1, x_3)$\hfill (K4)
        \item $R_1^2(x_1, x_3)$\hfill (3), (4), MP
        \item $\forall x_3 R_1^2(x_1, x_3)$\hfill (5), Gen
        \item $\forall x_1\forall x_3 R_1^2(x_1, x_3)$\hfill (6), Gen
        \item $\forall x_1\forall x_3 R_1^2(x_1, x_3)\to \forall x_3 R_1^2(x_2, x_3)$\hfill (K4)
        \item $\forall x_3 R_1^2(x_2, x_3)$\hfill (7), (8), MP
        \item $\forall x_2\forall x_3 R_1^2(x_2, x_3)$\hfill (9), Gen
    \end{enumerate}
\end{solution}
\end{document}
