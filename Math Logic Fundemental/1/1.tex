\documentclass[boxes]{homework}

% This is a slightly-more-than-minimal document that uses the homework class.
% See the README at http://git.io/vZWL0 for complete documentation.

\name{傅申 PB20000051}        % Replace (Your Name) with your name.
\term{2022 春}     % Replace (Current Term) with the current term.
\course{数理逻辑基础}    % Replace (Course Name) with the course name.
\hwnum{1}          % Replace (Number) with the number of the homework.
\hwname{作业}    
\problemname{习题}    
\solutionname{解:}

% Load any other packages you need here.
\usepackage[
    a4paper,
    top = 2.43cm,
    bottom = 2.43cm,
    left = 1.91cm,
    right = 1.91cm,
    includeheadfoot
]{geometry}
\fancyfootoffset{0pt} % make fancyhdr work properly
\usepackage{ctex}

\begin{document}

\begin{problem}
列出以下复合命题的真值表. (其中支命题 $p,q,r,s$ 视为问题变元.)
\begin{parts}
    \part
    \label{1.1}
    $(\lnot p\land q)\to (\lnot q\land r)$
    \part
    \label{1.2}
    $(p\to q)\to (p\to r)$
    \part
    \label{1.3}
    $\lnot (p\lor (q\land r))\leftrightarrow ((p\lor q)\land (p\lor r))$
\end{parts}
\end{problem}

\begin{solution}
\ref{1.1}
\begin{center}
    \begin{tabular}{cccc|c|cccc}
        $(\lnot$ & $p$ & $\land$ & $q)$ & $\to$ & $(\lnot$ & $q$ & $\land$ & $r)$\\
        \hline
        1 & 0 & 0 & 0 & 1 & 1 & 0 & 0 & 0\\
        1 & 0 & 0 & 0 & 1 & 1 & 0 & 1 & 1\\
        1 & 0 & 1 & 1 & 0 & 0 & 1 & 0 & 0\\
        1 & 0 & 1 & 1 & 0 & 0 & 1 & 0 & 1\\
        0 & 1 & 0 & 0 & 1 & 1 & 0 & 0 & 0\\
        0 & 1 & 0 & 0 & 1 & 1 & 0 & 1 & 1\\
        0 & 1 & 0 & 1 & 1 & 0 & 1 & 0 & 0\\
        0 & 1 & 0 & 1 & 1 & 0 & 1 & 0 & 1
    \end{tabular}
\end{center}
\ref{1.2}
\begin{center}
    \begin{tabular}{ccc|c|ccc}
        $(p$ & $\to$ & $q)$ & $\to$ & $(p$ & $\to$ & $r)$ \\
        \hline
        0 & 1 & 0 & 1 & 0 & 1 & 0\\
        0 & 1 & 0 & 1 & 0 & 1 & 1\\
        0 & 1 & 1 & 1 & 0 & 1 & 0\\
        0 & 1 & 1 & 1 & 0 & 1 & 1\\
        1 & 0 & 0 & 1 & 1 & 0 & 0\\
        1 & 0 & 0 & 1 & 1 & 1 & 1\\
        1 & 1 & 1 & 0 & 1 & 0 & 0\\
        1 & 1 & 1 & 1 & 1 & 1 & 1
    \end{tabular}
\end{center}
\ref{1.3}
\begin{center}
    \begin{tabular}{cccccc|c|ccccccc}
        $\lnot$ & $(p$ & $\lor$ & $(q$ & $\land$ & $r))$ & $\leftrightarrow$ & $((p$ & $\lor$ & $q)$ & $\land$ & $(p$ & $\lor$ & $r))$\\
        \hline
        1 & 0 & 0 & 0 & 0 & 0 & 0 & 0 & 0 & 0 & 0 & 0 & 0 & 0\\
        1 & 0 & 0 & 0 & 0 & 1 & 0 & 0 & 0 & 0 & 0 & 0 & 1 & 1\\
        1 & 0 & 0 & 1 & 0 & 0 & 0 & 0 & 1 & 1 & 0 & 0 & 0 & 0\\
        0 & 0 & 1 & 1 & 1 & 1 & 0 & 0 & 1 & 1 & 1 & 0 & 1 & 1\\
        0 & 1 & 1 & 0 & 0 & 0 & 0 & 1 & 1 & 0 & 1 & 1 & 1 & 0\\
        0 & 1 & 1 & 0 & 0 & 1 & 0 & 1 & 1 & 0 & 1 & 1 & 1 & 1\\
        0 & 1 & 1 & 1 & 0 & 0 & 0 & 1 & 1 & 1 & 1 & 1 & 1 & 0\\
        0 & 1 & 1 & 1 & 1 & 1 & 0 & 1 & 1 & 1 & 1 & 1 & 1 & 1\\
    \end{tabular}
\end{center}
\end{solution}

\begin{problem}
    写出由 $X_2=\{x_1, x_2\}$ 生成的公式集 $L(X_2)$ 的三个层次: $L_0$, $L_1$ 和 $L_2$.
\end{problem}
\begin{solution}
    \begin{gather}
        \begin{aligned}
            L_0=X_2=\{x_1, x_2\}
        \end{aligned}\\
        \begin{aligned}
            L_1=\{&\lnot x_1, \lnot x_2, x_1\to x_1, x_1 \to x_2, x_2\to x_1, x_2\to x_2\}
        \end{aligned}\\
        \begin{aligned}
            L_2=\{&\lnot(\lnot x_1), \lnot(\lnot x_2),\\
                  &\lnot(x_1\to x_1), \lnot(x_1\to x_2), \lnot(x_2\to x_1), \lnot(x_2\to x_2),\\
                  &x_1\to(\lnot x_1), x_1\to (\lnot x_2), x_2\to (\lnot x_1), x_2\to (\lnot x_2),\\
                  &(\lnot x_1)\to x_1, (\lnot x_1)\to x_2, (\lnot x_2)\to x_1, (\lnot x_2)\to x_2,\\
                  &x_1\to(x_1\to x_1), x_1\to(x_1\to x_2), x_1\to(x_2\to x_1), x_1\to(x_2\to x_2),\\
                  &x_2\to(x_1\to x_1), x_2\to(x_1\to x_2), x_2\to(x_2\to x_1), x_2\to(x_2\to x_2),\\
                  &(x_1\to x_1)\to x_1, (x_1\to x_2)\to x_1, (x_2\to x_1)\to x_1, (x_2\to x_2)\to x_1,\\
                  &(x_1\to x_1)\to x_2, (x_1\to x_2)\to x_2, (x_2\to x_1)\to x_2, (x_2\to x_2)\to x_2\}
        \end{aligned}
    \end{gather}
\end{solution}
\end{document}
