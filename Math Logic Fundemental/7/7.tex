\documentclass[boxes]{homework}

% This is a slightly-more-than-minimal document that uses the homework class.
% See the README at http://git.io/vZWL0 for complete documentation.

\name{傅申 PB20000051}        % Replace (Your Name) with your name.
\term{2022 春}     % Replace (Current Term) with the current term.
\course{数理逻辑基础}    % Replace (Course Name) with the course name.
\hwnum{7}          % Replace (Number) with the number of the homework.
\hwname{作业}    
\problemname{练习}    
\solutionname{解:}

% Load any other packages you need here.
\usepackage[
    a4paper,
    top = 2.54cm,
    bottom = 2.54cm,
    left = 1.91cm,
    right = 1.91cm,
    includeheadfoot
]{geometry}
\fancyfootoffset{0pt} % make fancyhdr work properly
\usepackage{ctex}
\setlength{\tabcolsep}{3pt} % Default value: 6pt
\renewcommand{\arraystretch}{1}

\begin{document}
\setlength\abovedisplayskip{.125em}
\setlength\belowdisplayskip{.125em}
\problemnumber{15}
\begin{problem}
4. 设 $x$ 不在 $p$ 中自由出现. 求证:
\begin{parts}[n]
    \item \label{15.4.1} $\vdash (p\to \forall xq)\to \forall x(p\to q)$
    \item \label{15.4.2} $\vdash (p\to \exists xq)\to \exists x(p\to q)$
\end{parts}
\end{problem}

\begin{solution}
    \ref{15.4.1} 要证明 $(p\to \forall xq)\to \forall x(p\to q)$, 只用证明 $\{p\to \forall xq\}\vdash \forall x(p\to q)$, 过程中除了 $x$ 以外不使用别的 Gen 变元, 如下:
    \begin{enumerate}[label = (\arabic*), parsep = 0pt, itemsep = 0pt, topsep = .25em]
        \item $p\to \forall xq$\hfill 假定
        \item $\forall xq\to q$\hfill (K4)
        \item $p\to q$\hfill (1), (2), HS
        \item $\forall x(p\to q)$\hfill (3), Gen
    \end{enumerate}
    因为 $x$ 不在 $p\to \forall xq$ 中自由出现, 由演绎定理, 不增加新的 Gen 变元就可得 $\vdash (p\to \forall q)\to \forall (p\to \forall q)$.
    \\
    \ref{15.4.2} 这里先证明 $\vdash \lnot (p\to q)\to p$ 和 $\vdash \lnot (p\to q)\to \lnot q$.
    \begin{enumerate}[parsep = 0pt, itemsep = .25em, topsep = .25em]
        \item 利用演绎定理和反证律, 以下公式从 $\{\lnot (p\to q), \lnot p\}$ 可证:
              \begin{enumerate}[label = (\arabic*), parsep = 0pt, itemsep = 0pt, topsep = .25em]
                  \item $\lnot p$\hfill 新假定
                  \item $\lnot p\to (p\to q)$\hfill 否定前件律
                  \item $p\to q$\hfill (1), (2), MP
                  \item $\lnot (p\to q)$\hfill 假定
              \end{enumerate}
              由 (3), (4) 用反证律可得 $\{\lnot(p\to q)\}\vdash p$, 再由演绎定理得到 $\vdash \lnot (p\to q)\to p$.
        \item 利用演绎定律和归谬律, 以下公式从 $\{\lnot(p\to q), q\}$ 可证:
              \begin{enumerate}[label = (\arabic*), parsep = 0pt, itemsep = 0pt, topsep = .25em]
                  \item $q$\hfill 新假定
                  \item $q\to (p\to q)$\hfill (L1)
                  \item $p\to q$\hfill (1), (2), MP
                  \item $\lnot (p\to q)$\hfill 假定
              \end{enumerate}
              由 (3), (4) 用归谬律可得 $\{\lnot(p\to q)\}\vdash \lnot q$, 再由演绎定理得到 $\vdash \lnot (p\to q)\to \lnot q$.
    \end{enumerate}

    然后证明题中命题, 为此只用证 $\{p\to\exists xq\}\vdash\exists x(p\to q)$, 过程中不使用除 $x$ 以外的 Gen 变元.

    以下公式从 $\{p\to\exists xq, \forall x\lnot(p\to q)\}$ 可证:
    \begin{enumerate}[label = (\arabic*), parsep = 0pt, itemsep = 0pt, topsep = .25em]
        \item $\forall x\lnot (p\to q)$\hfill 新假定
        \item $\forall x\lnot (p\to q)\to \lnot(p\to q)$\hfill (K4)
        \item $\lnot (p\to q)$\hfill (1), (2), MP
        \item $\lnot (p\to q)\to p$\hfill 已证明
        \item $p$\hfill (3), (4), HS
        \item $p\to \exists xq$\hfill 假定
        \item $\exists xq\ (=\lnot \forall x\lnot q)$\hfill (5), (6), MP
        \item $\lnot (p\to q)\to \lnot q$\hfill 已证明
        \item $\lnot q$\hfill (3), (8), MP
        \item $\forall x\lnot q$\hfill (9), Gen
    \end{enumerate}
    因为 $x$ 不在 $\forall x\lnot (p\to q)$ 中自由出现, 由 (9), (10) 用归谬律可得 $\{p\to \exists xq\}\vdash \exists x(p\to q)$, 再用演绎定理得到 $\vdash (p\to \exists xq)\to \exists x(p\to q)$.
\end{solution}
\problemnumber{16}
\begin{problem}
    1. 设 $x$ 不在 $q$ 中自由出现. 求证:
    \begin{parts}[n]
        \part \label{16.1.1} $\vdash (\exists xp\to q)\to \forall x(p\to q)$
        \part \label{16.1.2} $\vdash \exists x(p\to q)\to (\forall xp\to q)$
    \end{parts}
\end{problem}
\begin{solution}
    \ref{16.1.1} 要证 $\vdash (\exists xp\to q)\to \forall x(p\to q)$, 只用证 $\{\exists xp\to q\}\vdash \forall x(p\to q)$, 过程中不使用除 $x$ 以外的 Gen 变元.
    \begin{enumerate}[label = (\arabic*), parsep = 0pt, itemsep = 0pt, topsep = .25em]
        \item $\lnot \forall x \lnot p \to q$\hfill 假定
        \item $(\lnot \forall x \lnot p \to q)\to (\lnot q\to \lnot\lnot \forall x \lnot p)$\hfill 换位律
        \item $\lnot q\to \lnot\lnot \forall x \lnot p$\hfill (1), (2), MP
        \item $\lnot\lnot \forall x\lnot p\to \forall x\lnot p$\hfill 双重否定律
        \item $\lnot q\to \forall x\lnot p$\hfill (3), (4), HS
        \item $\forall x\lnot p\to \lnot p$\hfill (K4)
        \item $\lnot q\to \lnot p$\hfill (5), (6), HS
        \item $(\lnot q\to \lnot p)\to (p\to q)$\hfill (K3)
        \item $p\to q$\hfill (7), (8), MP
        \item $\forall x(p\to q)$\hfill (9), Gen
    \end{enumerate}
    因为 $x$ 不在 $\exists xp\to q$ 中自由出现, 由演绎定理, 不增加新的 Gen 变元就可得 $\vdash (\exists xp\to q)\to \forall x(p\to q)$.

    \ref{16.1.2} 要证 $\vdash \exists x(p\to q)\to (\forall xp\to q)$, 只用证 $\{\exists x(p\to q), \forall xp\}\vdash q$, 过程中不使用除 $x$ 以外的 Gen 变元.
    
    以下公式从 $\{\exists x(p\to q), \forall xp, \lnot q\}$ 可证:
    \begin{enumerate}[label = (\arabic*), parsep = 0pt, itemsep = 0pt, topsep = .25em]
        \item $\forall xp$\hfill 假定
        \item $\forall xp \to p$\hfill (K4)
        \item $p$\hfill (1), (2), MP
        \item $p\to (\lnot q\to \lnot (p\to q))$\hfill 永真式
        \item $\lnot q\to\lnot (p\to q)$\hfill (3), (4), MP
        \item $\lnot q$\hfill 新假定
        \item $\lnot (p\to q)$\hfill (6), (5), MP
        \item $\forall x\lnot (p\to q)$\hfill (7), Gen
        \item $\lnot \forall x\lnot (p\to q)$\hfill 假定
    \end{enumerate}
    其中式 (4) 的真值表见表 \ref{tab:16.1.2}, 因 Gen 变元 $x$ 不在假定中自由出现, 由 (8), (9) 用反证律得 $\{\exists x(p\to q), \forall xp\}\vdash q$, 再用两次演绎定理得 $\vdash \exists x(p\to q)\to (\forall xp\to q)$.
    \newpage
    \begin{table}[!htbp]
        \centering
        \caption{$p\to (\lnot q\to \lnot (p\to q))$ 的真值表}
        \label{tab:16.1.2}
        \begin{tabular}{c|c|ccccccc}
            $p$ & $\to$ & $(\lnot$ & $q$ & $\to$ & $\lnot$ & $(p$ & $\to$ & $q))$ \\
            \hline
            1 & 1 & 0 & 1 & 1 & 0 & 1 & 1 & 1\\
            1 & 1 & 1 & 0 & 1 & 1 & 1 & 0 & 0\\
            0 & 1 & 0 & 1 & 1 & 0 & 0 & 1 & 1\\
            0 & 1 & 1 & 0 & 0 & 0 & 0 & 1 & 0\\ 
        \end{tabular}
    \end{table}
\end{solution}
\problemnumber{16}
\begin{problem}
    3. 找出与所给公式等价的前束范式.
    \begin{parts}[n]
        \setcounter{enumi}{2}
        \part \label{16.3.3} $\forall x_1(R_1^1(x_1)\to R_1^2(x_1, x_2))\to (\exists x_2 R_1^1(x_2)\to \exists x_3R_1^3(x_2, x_3))$
        \part \label{16.3.4} $\exists x_1 R_1^2(x_1, x_2)\to (R_1^1(x_1)\to \lnot \exists x_3 R_1^2 (x_1, x_3))$
    \end{parts}
\end{problem}
\begin{solution}
    \ref{16.3.3} 适当改变题中公式的约束变元得到等价的 $q_1$:
    $$
        q_1 = \forall x_1(R_1^1(x_1)\to R_1^2(x_1, x_2))\to (\exists x_4 R_1^1(x_4)\to \exists x_3R_1^3(x_2, x_3))
    $$

    由 $q_1$ 出发, 得到以下的等价公式:

    $q_2=\exists x_1 ((R_1^1(x_1)\to R_1^2(x_1, x_2))\to (\exists x_4 R_1^1(x_4)\to \exists x_3R_1^3(x_2, x_3)))$\hfill (由命题 2-$2^\circ$)

    $q_3=\exists x_1 ((R_1^1(x_1)\to R_1^2(x_1, x_2))\to \exists x_3(\exists x_4 R_1^1(x_4)\to R_1^3(x_2, x_3)))$\hfill (由命题 2-$2^\circ$)

    $q_4=\exists x_1\exists x_3((R_1^1(x_1)\to R_1^2(x_1, x_2))\to(\exists x_4 R_1^1(x_4)\to R_1^3(x_2, x_3)))$\hfill (由命题 2-$2^\circ$)

    $q_5=\exists x_1\exists x_3((R_1^1(x_1)\to R_1^2(x_1, x_2))\to\forall x_4(R_1^1(x_4)\to R_1^3(x_2, x_3)))$\hfill (由命题 2-$2^\circ$)

    $q_6=\exists x_1\exists x_3\forall x_4((R_1^1(x_1)\to R_1^2(x_1, x_2))\to(R_1^1(x_4)\to R_1^3(x_2, x_3)))$\hfill (由命题 2-$2^\circ$)

    $q_6$ 即为所求的前束范式, 即
    $$
    \exists x_1\exists x_3\forall x_4((R_1^1(x_1)\to R_1^2(x_1, x_2))\to(R_1^1(x_4)\to R_1^3(x_2, x_3)))
    $$
    \\

    \ref{16.3.4}适当改变题中公式的约束变元得到等价的 $q_1$:
    $$
        q_1 = \exists x_4 R_1^2(x_4, x_2)\to (R_1^1(x_1)\to \lnot \exists x_3 R_1^2 (x_1, x_3))
    $$
    
    由 $q_1$ 出发, 得到以下的等价公式:

    $q_2=\forall x_4(R_1^2(x_4, x_2)\to (R_1^1(x_1)\to \lnot \exists x_3 R_1^2(x_1, x_3)))$\hfill (由命题 2-$2^\circ$)

    $q_3=\forall x_4(R_1^2(x_4, x_2)\to (R_1^1(x_1)\to \forall x_3 \lnot R_1^2(x_1, x_3)))$\hfill (由命题 2-$3^\circ$)

    $q_4=\forall x_4(R_1^2(x_4, x_2)\to \forall x_3 (R_1^1(x_1)\to \lnot R_1^2(x_1, x_3)))$\hfill (由命题 2-$2^\circ$)

    $q_5=\forall x_4\forall x_3 (R_1^2(x_4, x_2)\to (R_1^1(x_1)\to \lnot R_1^2(x_1, x_3)))$\hfill (由命题 2-$2^\circ$)

    $q_5$ 即为所求的前束范式, 利用 $\lnot(p\land (q\land r)) = p\to \lnot (q\land r)=p\to (q\to\lnot r)$, 得到所求的前束范式
    $$
        \forall x_4\forall x_3 \lnot(R_1^2(x_4, x_2)\land R_1^1(x_1)\land R_1^2(x_1, x_3))
    $$
\end{solution}
\end{document}
