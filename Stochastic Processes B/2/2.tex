\documentclass[boxes]{homework}

% This is a slightly-more-than-minimal document that uses the homework class.
% See the README at http://git.io/vZWL0 for complete documentation.

\name{傅申 PB20000051}        % Replace (Your Name) with your name.
\term{2022 春}     % Replace (Current Term) with the current term.
\course{随机过程 B}    % Replace (Course Name) with the course name.
\hwnum{2}          % Replace (Number) with the number of the homework.
\hwname{作业}    
\problemname{习题}    
\solutionname{解:}

% Load any other packages you need here.
\usepackage[
    a4paper,
    top = 2.54cm,
    bottom = 2.54cm,
    left = 1.91cm,
    right = 1.91cm,
    includeheadfoot
]{geometry}
\fancyfootoffset{0pt} % make fancyhdr work properly
\usepackage{ctex}

\begin{document}
\problemnumber{2}
\begin{problem}
$\{N(t), t\geq 0\}$ 为一强度是 $\lambda$ 的 Poisson 过程. 对 $s > 0$ 试计算 $E[N(t)\cdot N(t+s)]$.
\end{problem}

\begin{solution}
我们对 $N(t)\cdot N(t+s)$ 稍作变换
\begin{equation}
    N(t)\cdot N(t+s) = N(t)\cdot [(N(t+s) - N(t)) + N(t)] = (N(t)-N(0))^2 + (N(t)-N(0))\cdot (N(t+s)-N(t))
\end{equation}
其中 $N(t)-N(0)$ 与 $N(t+s)-N(t)$ 相互独立. 而 $(N(t)-N(0))\sim \mathrm{Poisson}(\lambda t)$, 则有
\begin{equation}
    \begin{aligned}
        E[N(t)\cdot N(t+s)] &= E[(N(t)-N(0))^2] + E[(N(t)-N(0))\cdot (N(t+s)-N(t))] \\
        &= \mathrm{Var}[N(t)-N(0)] + E^2[N(t)-N(0)] + E[N(t)-N(0)]\cdot E[N(t+s)-N(t)] \\
        &= \lambda t + (\lambda t)^2 + \lambda t\cdot \lambda s \\
        &= \lambda t [1 + \lambda(t+s)]
    \end{aligned}
\end{equation}
\end{solution}

\problemnumber{4}
\begin{problem}
    $\{N(t), t\geq 0\}$ 为一 $\lambda = 2$ 的 Poisson 过程, 试求:
    \begin{parts}[r]
        \part
        \label{4.1}
        $P\{N(1)\leq 2\}$
        \part
        \label{4.2}
        $P\{N(1) = 1\text{ 且 }\ N(2) = 3\}$
        \part
        \label{4.3}
        $P\{N(1)\geq 2\vert N(1)\geq 1\}$
    \end{parts}
\end{problem}
\begin{solution}
    \ref{4.1}
    \begin{equation}
        \begin{aligned}
            P\{N(1)\leq 2\} &= \sum_{k=0}^{2} \frac{(\lambda)^k\exp(-\lambda)}{k!} \\
            &= \left(\frac{1}{1}+\frac{2}{1}+\frac{4}{2}\right)\exp(-2)\\
            &= 5\mathrm{e}^{-2}\approx 0.6767
        \end{aligned}
    \end{equation}
    \ref{4.2} $N(2)-N(1)$ 与 $N(1)$ 相互独立, 有
    \begin{equation}
        \begin{aligned}
            P\{N(1) = 1\text{ 且 }\ N(2) = 3\} &= P\{N(1) = 3\}\cdot P\{N(2) - N(1) = 2\vert N(1) = 3\}\\
            &= \mathrm{e}^{-2}\cdot P\{N(2) - N(1) = 2\}\\
            &= 2\mathrm{e}^{-2}\cdot 2\mathrm{e}^{-2}\\
            &= 4\mathrm{e}^{-4}\approx 0.0733
        \end{aligned}
    \end{equation}
    \ref{4.3} 显然 $P\{N(1)\ge 2, N(1)\ge 1\} = P\{N(1)\geq 2\}$, 有
    \begin{equation}
        P\{N(1)\geq 2\vert N(1)\geq 1\}=\frac{P\{N(1)\geq 2\}}{P\{N(1)\geq 1\}}=\frac{1-3\mathrm{e}^{-2}}{1-\mathrm{e}^{-2}}\approx 0.6870
    \end{equation}
\end{solution}
\problemnumber{7}
\begin{problem}
    $N(t)$ 是强度为 $\lambda$ 的 Poisson 过程. 给定 $N(t) = n$, 试求第 $r$ 个事件 ($r\leq n$) 发生的时刻 $W_r$ 的条件概率密度 $f_{W_r\vert N(t)=n}(w_r\vert n)$.
\end{problem}
\begin{solution}
    给定 $N(t) = n$, 不妨设 $W_r = w_r$. 对充分小的增量 $\Delta w_r\downarrow0$, 有
    \begin{equation}
        \begin{aligned}
            &\{w_r\leq W_r\leq w_r+\Delta w_r, N(t) = n\} \\
            &= \{N(w_r) = r-1, N(w_r+\Delta w_r)-N(w_r) = 1, N(t) - N(t - w_r - \Delta w_r) = n - r\}
        \end{aligned}
    \end{equation}
    记 $P_1 = P(N(w_r) = r-1, N(w_r+\Delta w_r)-N(w_r) = 1, N(t) - N(t - w_r - \Delta w_r) = n - r)$, 于是
    \begin{equation}
        \begin{aligned}
            f_{W_r\vert N(t)=n}(w_r\vert n)\Delta w_r &= P(w_r\leq W_r< w_r + \Delta w_r\vert N(t)=n) + o(\Delta w_r)\\
            &= \frac{P_1}{P(N(t) = n)} + o(\Delta w_r)
        \end{aligned}
    \end{equation}
    由独立增量性和 Poisson 过程的定义, 有
    \begin{gather}
        \begin{aligned}
            P_1 &= P(N(w_r) = r-1, N(w_r+\Delta w_r)-N(w_r) = 1, N(t) - N(t - w_r - \Delta w_r) = n - r)\\
            &= \frac{(\lambda w_r)^{r - 1}\exp(-\lambda w_r)}{(r - 1)!}(\lambda \Delta w_r + o(\Delta w_r))\frac{(\lambda (t - w_r - \Delta w_r))^{n - r}\exp(-\lambda (t - w_r - \Delta w_r))}{(n - r)!}\\
            &= \frac{\lambda^n}{(r-1)!(n-r)!}w_r^{r-1}(t - w_r - \Delta w_r)^{n - r}\exp(-\lambda t)\exp(\lambda \Delta w_r)\Delta w_r + o(\Delta w_r)
        \end{aligned}\\
        P(N(t) = n) = \frac{(\lambda t)^n\exp(-\lambda t)}{n!}
    \end{gather}
    因此, 得到
    \begin{equation}
        \begin{aligned}
            f_{W_r\vert N(t)=n}(w_r\vert n) &= \lim_{\Delta w_r\downarrow 0} \frac{P_1}{P(N(t) = n)\Delta w_r}\\
            &= \lim_{\Delta w_r\downarrow 0} \frac{n!}{(r-1)!(n-r)!}\frac{w_r^{r-1}(t - w_r - \Delta w_r)^{n - r}}{t^n}\exp(\lambda \Delta w_r)\\
            &= \frac{n!}{(r-1)!(n-r)!}\frac{w_r^{r-1}(t - w_r)^{n - r}}{t^n}
        \end{aligned}
    \end{equation}
\end{solution}
\begin{problem}
    令 $\{N_i(t), t\geq 0\}, i = 1,2,\cdots, n$ 为 $n$ 个独立的有相同强度参数为 $\lambda$ 的 Poisson 过程. 记 $T$ 为在全部 $n$ 个过程中至少发生了一件事的时刻, 试求 $T$ 的分布.
\end{problem}
\begin{solution}
    由题意可知
    \begin{equation}
        \{T>t\} = \{N_i(t) = 0, i = 1, 2, \cdots, n\}
    \end{equation}
    因此 
    \begin{equation}
        P(T>t) = \prod_{i = 1}^n P(N_i(t) = 0) = \prod_{i = 1}^n \exp(-\lambda t) = \exp(-n\lambda t)
    \end{equation}
    因此有
    \begin{gather}
        F_T(t) = 1 - \exp(-n\lambda t)\\
        f_T(t) = n\lambda \exp(-n\lambda t)
    \end{gather}
\end{solution}
\problemnumber{10}
\begin{problem}
    到达某加油站的公路上的卡车服从参数为 $\lambda_1$ 的 Poisson 过程 $N_1(t)$, 而到达的小汽车服从 $\lambda_2$ 的 Poisson 过程 $N_2(t)$, 且过程 $N_1$ 与 $N_2$ 独立. 试问随机过程 $N(t)=N_1(t)+N_2(t)$ 是什么过程? 并计算在总车流数 $N(t)$ 中卡车首先到达的概率.
\end{problem}
\begin{solution}
    首先, $N(0)=N_1(0)+N_2(0)=0$.

    其次, 因为 $N_1(t)$ 与 $N_2(t)$ 独立, 所以对任意的 $0\leq t_1<\cdots<t_n$, $\{N_1(t_2) - N_1(t_1),\cdots,N_1(t_n)-N_1(t_{n-1})\}$ 与 $\{N_2(t_2) - N_2(t_1),\cdots,N_2(t_n)-N_2(t_{n-1})\}$ 相互独立, 且它们内部也相互独立, 所以 $\{N(t_2) - N(t_1),\cdots,N(t_n)-N(t_{n-1})\}$ 也相互独立, $N(t)$ 是独立增量过程.

    最后, 有\vspace*{-1em}
    \begin{align}
        \begin{aligned}
            P\{N(s+t) - N(s)=k\} &= \sum_{i = 0}^k P\{N_1(s+t) - N_1(s)=i\}P\{N_2(s+t) - N_2(s)=k-i\}\\
            &= \sum_{i = 0}^k \frac{(\lambda_1t)^i(\lambda_2t)^{k-i}\exp[-(\lambda_1+\lambda_2)t]}{i!(k-i)!}\\
            &= \frac{t^k\exp[-(\lambda_1+\lambda_2)t]}{k!}\sum_{i = 0}^k \mathrm{C}_k^i\lambda_1^i\lambda_2^{k-i}\\
            &= \frac{[(\lambda_1 + \lambda_2)t]^k\exp[-(\lambda_1+\lambda_2)t]}{k!}
        \end{aligned}
    \end{align}
    
    所以 $N(t)$ 是参数为 $\lambda_1+\lambda_2$ 的 Poisson 过程.

    记 $X, Y$ 分别为过程 $N_1(t)$ 和 $N_2(t)$ 中第一辆车到达的时刻, 则 $X$, $Y$ 独立, 且分别是均值为 $\displaystyle \frac{1}{\lambda_1},\frac{1}{\lambda_2}$ 的指数分布随机变量. $\{X<Y\}$ 对应卡车首先到达, 其概率为 
    \begin{align}
        \begin{aligned}
            P\{X<Y\}&=\iint_{x,y>0, x<y}\lambda_1\exp(-\lambda_1x)\lambda_2\exp(-\lambda_2y)\mathrm{d}x\mathrm{d}y\\
            &=\lambda_1\int_{0}^\infty\exp(-\lambda_1x)\int_x^\infty\lambda_2\exp(-\lambda_2y)\mathrm{d}y\mathrm{d}x\\
            &=\lambda_1\int_0^\infty\exp[-(\lambda_1+\lambda_2)x]\mathrm{d}x\\
            &=\frac{\lambda_1}{\lambda_1+\lambda_2}
        \end{aligned}
    \end{align}
\end{solution}
\begin{problem}
    冲击模型 (Shock Model) \quad 记 $N(t)$ 为某系统某时刻 $t$ 收到的冲击次数, 它是参数为 $\lambda$ 的 Poisson 过程. 设第 $k$ 次冲击对系统的损害大小 $Y_k$ 服从参数为 $\mu$ 的指数分布, $Y_k, k=1,2,\cdots$ 独立同分布. 记 $X(t)$ 为系统所受到的总损害. 当损害超过一定极限 $\alpha$ 时系统不能运行, 寿命终止, 记 $T$ 为系统寿命. 试求该系统的平均寿命 $ET$, 并对所得结果做出直观解释.
\end{problem}
\begin{solution}
    显然 $X(t)$ 是一个复合 Poisson 过程, 有 $\displaystyle X(t) = \sum_{k = 1}^{N(t)}Y_k$. 而事件 $\{T>t\}$ 对应 $\{X(t)\leq\alpha\}$. 同时 $Y_k$ 与 $N(t)$ 相互独立, 有
    \begin{equation}
        P(T>t)=P\left(\sum_{k = 1}^{N(t)}Y_k\leq\alpha\right)=\sum_{n=1}^\infty P\left(\sum_{k = 1}^{n}Y_k\leq\alpha\right)P(N(t)=n)
    \end{equation}
    记 $S_n=\displaystyle\sum_{k = 1}^{n}Y_k$, 则 $S_n\sim \Gamma(n,\mu)$, 即
    \begin{equation}
        f_{S_n}(s) = \frac{\mu^n}{(n-1)!}s^{n-1}\exp(-\mu s)
    \end{equation}
    因此有
    \begin{equation}
        P\left(\sum_{k = 1}^{n}Y_k\leq\alpha\right) = \frac{\mu^n}{(n-1)!}\int_0^\alpha s^{n-1}\exp(-\mu s)\mathrm{d}s
    \end{equation}
    所以
    \begin{gather}
        P(T>t) = \exp(-\lambda t)\sum_{n = 1}^\infty \frac{(\mu\lambda t)^n}{n!(n-1)!}\int_0^\alpha s^{n-1}\exp(-\mu s)\mathrm{d}s\\
        \begin{aligned}
            E(T) &= \int_0^\infty P(T>t)\mathrm{d}t
            = \int_0^\infty \exp(-\lambda t)\sum_{n = 1}^\infty \frac{(\mu\lambda t)^n}{n!(n-1)!}\int_0^\alpha s^{n-1}\exp(-\mu s)\mathrm{d}s\mathrm{d}t\\
            &= \sum_{n = 1}^\infty \frac{\mu^n}{n!(n-1)!}\int_0^\infty (\lambda t)^n\exp(-\lambda t)\mathrm{d}t\int_0^\alpha s^{n-1}\exp(-\mu s)\mathrm{d}s\\
            &= \sum_{n=1}^\infty\frac{\mu^n}{n!(n-1)!}\frac{n!}{\lambda}\int_0^\alpha s^{n-1}\exp(-\mu s)\mathrm{d}s\\
            &= \frac{1}{\lambda}\int_0^\alpha\exp(-\mu s)\sum_{n=1}^\infty\frac{\mu^n}{(n-1)!}s^{n-1}\mathrm{d}s\\
            &= \frac{1}{\lambda}\int_0^\alpha \mu\mathrm{d}s\\
            &= \frac{\mu\alpha}{\lambda}
        \end{aligned}
    \end{gather}
    直观解释: 若冲击越频繁 ($\lambda$ 越大), 每次冲击造成的平均伤害越大 ($\mu$ 越小), 系统能承受的伤害极限越小 ($\alpha$ 越小), 则系统寿命平均越短.
\end{solution}
\begin{problem}
    令 $N(t)$ 是强度函数为 $\lambda(t)$ 的非齐次 Poisson 过程, $X_1, X_2, \cdots$ 为事件间的时间间隔.
    \begin{parts}[r]
        \part
        \label{12.1}
        $X_i$ 是否独立
        \part
        \label{12.2}
        $X_i$ 是否同分布
        \part
        \label{12.3}
        试求 $X_1$ 及 $X_2$ 的分布函数
    \end{parts}
\end{problem}
\begin{solution}
    \ref{12.1} $X_i$ 不独立, 例如对 $X_1$ 有
    \begin{equation}\label{eq:12.1}
        P(X_1>t)=P(N(t)=0) = \exp\left(-\int_0^t\lambda(u)\mathrm{d}u\right)
    \end{equation}
    对 $X_2$ 有
    \begin{equation}
        P(X_2>t\vert X_1 = s) = P(N(t+s)-N(s) = 0\vert X_1 = s) = P(N(s+t)-N(s))=\exp\left(-\int_s^{s+t}\lambda(u)\mathrm{d}u\right)
    \end{equation}
    即 $P(X_2|X_1)$ 与 $X_1$ 的取值有关, 则 $X_1$ 和 $X_2$ 不独立.
    \newpage
    \ref{12.2} $X_i$ 不同分布. 例如 $X_1$ 的分布如式 (\ref{eq:12.1}) 所示, 其密度函数为
    \begin{equation}\label{eq:12.3}
        f_{X_1}(x) = \exp\left(-\int_0^x\lambda(u)\mathrm{d}u\right)\lambda(x)
    \end{equation}
    而 $X_2$ 的分布如下
    \begin{equation}\label{eq:12.4}
        \begin{aligned}
            P(X_2>t) &= \int_0^\infty P(X_2>t\vert X_1=x)f_{X_1}(x)\mathrm{d}x=\int_0^\infty\exp\left(-\int_0^{x+t}\lambda(u)\mathrm{d}u\right)\lambda(x)\mathrm{d}x\\
            &= P(X_1>t)\int_0^\infty\exp\left(-\int_x^{x+t}\lambda(u)\mathrm{d}u\right)\lambda(x)\mathrm{d}x\neq P(X_1>t)
        \end{aligned}
    \end{equation}
    所以 $X_1$ 与 $X_2$ 不同分布.

    \ref{12.3} 由式 (\ref{eq:12.1}), (\ref{eq:12.3}), (\ref{eq:12.4}) 可得 $X_1$ 与 $X_2$ 的分布如下.
    \begin{gather}
        F_{X_1}(t) = 1-\exp\left(-\int_0^t\lambda(u)\mathrm{d}u\right)\\
        f_{X_1}(t) = \lambda(t)\exp\left(-\int_0^t\lambda(u)\mathrm{d}u\right)\\
        F_{X_2}(t) = 1-\int_0^\infty\lambda(s)\exp\left(-\int_0^{s+t}\lambda(u)\mathrm{d}u\right)\mathrm{d}s\\
        f_{X_2}(t) = \int_0^\infty\lambda(s)\lambda(s+t)\exp\left(-\int_0^{s+t}\lambda(u)\mathrm{d}u\right)\mathrm{d}s
    \end{gather}
\end{solution}
\end{document}
