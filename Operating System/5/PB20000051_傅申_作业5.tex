\documentclass[11pt]{homework}

% TODO: replace these with your information
\newcommand{\hwname}{傅申}
\newcommand{\hwid}{PB20000051}
\newcommand{\hwtype}{操作系统作业}
\newcommand{\hwnum}{5}
\usepackage{calc}
\usepackage{enumitem}

\begin{document}
\setlength{\intextsep}{0em}
\setlength{\belowdisplayskip}{.25em} \setlength{\belowdisplayshortskip}{.25em}
\setlength{\abovedisplayskip}{.25em} \setlength{\abovedisplayshortskip}{.25em}
\maketitle
% Your content
\question
\begin{alphaparts}
    \questionpart Suppose block $A_1$ will be updated
    \begin{description}[
            leftmargin = !,
            labelwidth = \widthof{\bfseries RAID-5},
            topsep = 0pt,
            itemsep = 0pt,
            parsep = 0pt
        ]
        \item[RAID-5] For RMW, block $A_1$ and $A_p$ will be updated, \underline{2 blocks} will be
            accessed.

            But for RWM, in addition to $A_1$ and $A_p$, block $A_2$, $A_3$ and $A_4$ will
            be read to calculate the new parity, so \underline{5 blocks} will be accessed.
        \item[RAID-6] For RMW, block $A_1$, $A_p$ and $A_q$ will be updated, \underline{3 blocks}
            will be accessed.

            But for RWM, in addition to $A_1$, $A_p$ and $A_q$, block $A_2$ and $A_3$ will
            be read to calculate the new parity, so \underline{5 blocks} will be accessed.
    \end{description}
    \questionpart \hfill
    \begin{description}[
            leftmargin = !,
            labelwidth = \widthof{\bfseries RAID-5},
            topsep = 0pt,
            itemsep = 0pt,
            parsep = 0pt
        ]
        \item[RAID-5] Suppose $A_1$, $A_2$, $A_3$, $A_4$, $B_1$, $B_2$ and $B_3$ will be updated.

            For RMW, since the update of parity needs the old data, every data block update would
            cause the parity block to be updated, thus, $2\times 7$ = \underline{14 blocks} will be
            accessed.

            But for RRW, the update of parity only needs the new data, so the parity block update
            can be done after all the data block updates. In this case, $B_4$ will be read to
            calculate the new parity, thus, \underline{10 blocks} (7 updated block + $B_4$ + 2
            parity block) will be accessed.
        \item[RAID-6] Suppose $A_1$, $A_2$, $A_3$, $B_1$, $B_2$, $B_3$ and $C_1$ will be updated.

            For RMW, since the update of parity needs the old data, every data block update would
            cause the 2 parity blocks to be updated, thus, $3\times 7$ = \underline{21 blocks} will
            be accessed.

            But for RRW, the update of parity only needs the new data, so the parity block update
            can be done after all the data block updates. In this case, $C_1$ and $C_2$ will be read
            to calculate the new $C_p$, thus, \underline{15 blocks} (7 updated block + $C_2$ + $C_3$
            + 6 parity block) will be accessed.
    \end{description}
\end{alphaparts}
\question
\begin{description}[
        leftmargin = !,
        labelwidth = \widthof{\bfseries C-SCAN},
        topsep = 0pt,
        itemsep = 0pt,
        parsep = 0pt
    ]
    \item[FCFS] Order: 2150, 2069, 1212, 2296, 2800, 544, 1618, 356, 1523, 4965, 3681;

        Total distance: 13011 cylinders;

    \item[SSTF] Order: 2150, 2069, 2296, 2800, 3681, 4965, 1618, 1523, 1212, 544, 356;

        Total distance: 7586 cylinders;

    \item[SCAN] Order: 2150, 2296, 2800, 3681, 4965, (4999), 2069, 1618, 1523, 1212, 544, 356;

        Total distance: 7492 cylinders;

    \item[LOOK] Order: 2150,2296, 2800, 3681, 4965, 2069, 1618, 1523, 1212, 544, 356;

        Total distance: 7424 cylinders;

    \item[C-SCAN] Order: 2150, 2296, 2800, 3681, 4965, (4999, 0), 356, 544, 1212, 1523, 1618, 2069;

        Total distance: 9917 cylinders;

    \item[C-LOOK] Order: 2150, 2296, 2800, 3681, 4965, 356, 544, 1212, 1523, 1618, 2069;

        Total distance: 9137 cylinders;
\end{description}
\question
Most of file operations involve searching the directory for locating the file. To avoid this
constant searching, many system require that an \texttt{open()} system call be made before a file
is first used. The operating system keeps a table, called the \textbf{open-file table}, containing
information about all open files. When a file operation is requested, the file is specified via an
index into this table, so no searching is required. When the file is no longer being used, it is
closed by the process, and the operating system removes its entry from the open-file table,
potentially releasing locks.

With the open-file table, the open files is indexed in memory, easily accessible by the process.
We can get the open file from the open-file table rather than directory traversal for each access.
Considering that we usually have a sequential operation to the files, the open-file table can
improve performance and reduce I/O.

\question
\texttt{755} stands for \texttt{rwxr-xr-x}, meaning
\begin{itemize}
    \item The file's owner can read, write and execute the file.
    \item The file's group and others can read and execute the file.
\end{itemize}

\question
Suppose we need to write a new large file to the file system that we have enough space for. But
there is no hole big enough to fit the new file. The problem called external fragmentation occurs.
To solve this problem, we need the very expensive defragmentation process to reorganize the file system, moving
files together to make space for the new file.

Another problem is that the file may not have enough space to grow. To solve this problem, we need
to relocate the file to a new location that has enough space.

\question
The advantage:
\begin{itemize}
    \item External fragmentation problem is solved.
    \item Files can grow and shrink freely.
    \item Free block management is easy to implement.
    \item The random access problem can be eased by keeping a cached version of FAT inside the kernel.
\end{itemize}

The major problem is that the entire FAT has to be stored in memory so that the performance of
looking up of an arbitrary block is satisfactory.

\question
The read order is:
\begin{enumerate}
    \item Read the root directory.
    \item Read the index node of \texttt{/a}.
    \item Read the disk block of \texttt{/a}.
    \item Read the index node of \texttt{/a/b}.
    \item Read the disk block of \texttt{/a/b}.
    \item Read the index node of \texttt{/a/b/c}.
    \item Read the disk block of \texttt{/a/b/c}.
\end{enumerate}
\begin{alphaparts}
    \questionpart All read operation will invoke an I/O operation, thus, 7 I/O operation is required.
    \questionpart The inodes are in the memory, so reading them won't cause I/O operation. 4 I/O
    operation is required.
\end{alphaparts}
\question
The block size is 8 KB, i.e., $2^{13}$ bytes. A pointer requires $2^2$ bytes. The maximum file size
would be
$$
    \begin{aligned}
                                                       & \underbrace{12\times 2^{13}}_{\text{Direct Block}} & +                                            & \underbrace{1\times 2^{13-2}
        \times 2^{13}}_{\text{Indirect Block}}         & +                                                  & \underbrace{1\times 2^{13-2} \times 2^{13-2}
        \times 2^{13}}_{\text{Double Indirect Blocks}} & +                                                  & \underbrace{1\times 2^{13-2}
        \times 2^{13-2} \times 2^{13-2}\times 2^{12}}_{\text{Triple Indirect Blocks}}                                                                                                                               \\
        =                                              & 12 \times 2^{13}                                   & +                                            & 2^{24}                       & + & 2^{35} & + & 2^{46}
    \end{aligned}
$$
The result is 70,403,120,791,552 bytes, i.e., approximately 64 TB.

\question
A hard link is a directory entry pointing to an existing file. When it is created, there is no new
file content created, and the link count of the target file will be incremented. It's like a new
pathname of the target file is created.

A symbolic link is a file (shortcut) storing the pathname of the target file. When it is created,
there is a new inode created, and the link count of the target file won't change.

\question
The difference between data journaling and metadata journaling is that in data journaling, the data
will be written to disk twice, once to the journal and once to the on-disk location, but in
metadata journaling, the data will be written to disk only once, to the on-disk location only.

The sequence for data journaling is:
\begin{enumerate}
    \item Journal write: Write the contents of the transaction (including TxB, metadata, and data)
    \item Journal commit: metadata, and data (including TxE)
    \item Checkpoint: Write the contents of the update to their on-disk locations
    \item Free: free the space of log
\end{enumerate}

The sequence for metadata journaling is:
\begin{enumerate}
    \item Data write \& Journal metadata write: Write the contents of the transaction (including TxB
          and data) and write the contens of the on-disk locations. Two write can be issued in parallel.
    \item Journal commit: Write TxE
    \item Checkpoint: Write the contents of the update to their on-disk locations
    \item Free: free the space of log
\end{enumerate}

\question
Polling, Interrupt and Direct Memory Access (DMA).

\question
\begin{itemize}
    \item I/O scheduling
          \begin{itemize}
              \item Maintain a per-device queue
              \item Re-ordering the requests
              \item Average waiting time, fairness, etc.
          \end{itemize}
    \item Buffering - store data in memory while transferring between devices
          \begin{itemize}
              \item To cope with device speed mismatch
              \item To cope with device transfer size mismatch
              \item To maintain ``copy semantics'' (e.g., copy from application's buffer to kernel buffer)
          \end{itemize}
    \item Caching - faster device holding copy of data
          \begin{itemize}
              \item Always just a copy
              \item Key to performance
              \item Sometimes combined with buffering
          \end{itemize}
    \item Spooling - hold output for a device

          If device can serve only one request at a time, e.g., Printing

    \item Error handling and I/O protection
          \begin{itemize}
              \item OS can recover from disk read error, device unavailable, transient write failures
              \item All I/O instructions defined to be privileged
          \end{itemize}
    \item Power management, etc.
\end{itemize}
\end{document}
